\documentclass{article}
\title{The Opensource Club Constitution}
\author{Isaac Jones -- Martin Jansche -- Michael Benedict}
\date{20 March, 2000\\ Updated 02 September 2013}
\setcounter{secnumdepth}{0}
\usepackage[normalem]{ulem}
\begin{document}
	\maketitle

	\section{Preamble}

%	\sout{It is important to recognize the danger that bureaucracy can get in the way of doing cool things. To that end, this constitution is framed, and should remain as only a loosely governing tool. If the bureaucracy of the opensource club ever becomes a burden more than a tool, then the members should at that time, re-frame this document to fix the situation.}

	It is important to recognize the danger that bureaucracy can get in the way of doing cool things.  So long as the requirements of the Ohio State University and the realistic needs of the club are met, efforts should be made to minimize bureaucracy.

	As the club has evolved it has become necessary to elect additional officers in order to aid the further growth of the club. While this is not an attempt to create a bureaucracy, it is necessary in order to delegate duties among members in order to help the club run smoothly. At the time of this writing there are currently three elected positions within the club: President (or Benevolent Dictator), Vice President, and Treasurer. These three positions will continue to exist as long as they are required by the Ohio State University.

	\section{Article I - Name, Purpose and Non-Discrimination Policy}

	\subsection{Section 1 - Name}

	This is the Opensource Club at the Ohio State University.

	\subsection{Section 2 - Purpose}

	Our purpose is to write and advocate free software, and to create a community of excellent programmers. We advocate free software through creation of documentation, providing support, and fighting cluelessness.

	The club should take no ``official'' stand on political issues in debate among the opensource community, should adopt no ``official club license,'' should not create strict guidelines for what kind of software should be developed, should not create strict guidelines for how a software project should be managed (including frequency of updates, release models, code repository management, etc). Doing so would violate the spirit of the Preamble. Individual software projects can be managed however the individuals involved in the project see fit.

	Implicit in this is that the club must allow advocacy of proprietary software written for opensource platforms (for instance) or writing of open source software for a proprietary platform (for instance) if some of its members wish to do so. However, programs written by members of the club, under the flag of the club, and distributed by the club should meet some acceptable definition of open source software.

	\subsection{Section 3 - Non Discrimination Policy}

	In recognition of the importance of welcoming diversity for the sake of creativity, and for the benefit of humanity, this club welcomes all people. This means that our club and it's members will not discriminate against any individual(s) for reasons of age, color, disability, gender identity or expression, national origin, race, religion, sex, sexual orientation, or veteran status.

	\section{Article II - Membership: Qualifications and categories of membership}

%	\sout{Ohio State University guidelines demand that voting membership be limited to currently enrolled students. The current model of membership is that all people who attend meetings or are on our mailing list are considered members. This kind of loose definition should be preserved. No membership fee for joining the club should be charged.}

%	\sout{If the distinction between voting members and other members becomes a significant source of debate, that is a good indication that the club should re-draft its constitution because bureaucratic things are starting to become a burden.}

%	\sout{Voting membership is defined by attendance at required meetings. Voting members are required to attend required meetings.}

%	\sout{As the resources of the club grow and the volume of its membership expands, it becomes necessary to further define membership within the club. Due to the high turnover of membership that is inherent to organizations of this nature, there must be placed some restriction on voting membership in order to avoid difficult situations. In order for a member to obtain voting rights, said member must be in attendance at a reasonable amount of meetings or invest a significant amount of time or resources into the club. These criterion will be evaluated by the officers at the time of a vote. Computer accounts will be given out to any member who requests an account, provided that SOURCE rules are followed. These accounts may be revoked due to inactivity or failure to follow the rules. Key card access to the club office will be granted after a member has been active within the club for at least two months, provided that SOURCE  rules are followed. Access may be revoked due to inactivity or failure to follow the rules.}

	\subsection{Part 1 - Membership Categories and Selection Processes}
	Ohio State University guidelines demand that voting membership be limited to currently enrolled students.  In order to meet this requirement, membership will be divided into voting and non-voting categories.

	\begin{itemize}
		\item Non-voting membership can be obtained simply by attendance of meetings or presence on our mailing list.
		\item Voting membership requires that the individual meet all Ohio State University guidelines for voting membership, and that the individual has attended at least one meeting prior to the vote.
	\end{itemize}
	\subsection{Part 2 - Membership Privileges}

	Accounts on the club computers will be given out to any member who requests an account, provided Ohio State University's rules are followed.  These accounts may be revoked due to inactivity or failure to follow the rules.  Key card access to the club office will be granted after a member has been active within the club for at least two months, provided that Ohio State University rules are followed.  Access may be revoked due to inactivity or failure to follow the rules.
	When it is necessary to take a vote, voting members are allowed to do so.

	\subsection{Part 3 - Membership Removal}

	Should cause for membership removal be raised, a member can be removed in one of two ways:
	\begin{enumerate}
		\item A majority vote of club officers.
		\item A majority vote of voting members, providing at least there are at least 3/4\textsuperscript{th} of the usual number of attending voting members present at the vote.
	\end{enumerate}

	\section{Article III - Officer Positions, Duties, Selection, and Removal}

%	\sout{The club should elect one leader whose title shall be Benevolent Dictator. The traditional role of leadership is hereby rejected in favor of the evolving model of this club. The Benevolent Dictator shall be in office for one academic year. They should be elected because they are respected by the club members and because of his or her technical ability and knowledge of computers. The method of electing the Benevolent Dictator is by vote with a simple majority, where a quorum (3/4 of the voting membership) is present.}

%	\sout{The Benevolent Dictator will probably be the hardest working member of the club, and therefore needs the power to make unilateral decisions for the sake of saving time. The Benevolent Dictator will not necessarily have any control over software projects, since by the definition of opensource software, code forks can happen when they need to, and no one therefore has ultimate power over a software project.}

%	\sout{A system of checks and balances is implicit in the fact that all people are free to do as they please. If the Benevolent Dictator becomes unreasonable, he or she will cease to be Benevolent, and therefore lose their power.}

%	\sout{The club should have a treasurer, and will need technical assistance in maintaining computers, web pages, code repositories, and servers for the operation of the club. The club will need help in organizing its activities so that the Benevolent Dictator needn't do all of the work. Positions of this sort should be on a volunteer basis with appointment by the Benevolent Dictator where necessary.}

%	\sout{All officers, volunteers, and the Benevolent Dictator are servants of the club.}

	\subsection{Part 1 - Officer Positions, Duties, Powers and Limitations.}

	There are three officer positions: The President (or Benevolent Dictator), the Vice President, and the Treasurer.  These three positions will continue to exist as long as they are required by the Ohio State University.  The responsibilities of the aforementioned officers are as follows:

	The President shall have the following responsibilities:

	\begin{itemize}
		\item Managing key card access to the office
		\item Assignment of administrative accounts on the club computers
		\item Securing the meeting location
		\item Announcing meeting location and topic
		\item Representing the club at other functions and in public
		\item Appointment of miscellaneous positions
	\end{itemize}

	The Vice President shall have the following responsibilities:
	\begin{itemize}
		\item Keeping the minutes of the meeting
		\item Carrying out the duties of the President when the President is unable to do so
	\end{itemize}

	The Treasurer shall the following responsibilities:

	\begin{itemize}
		\item Produces and presents a quarterly budget
		\item Purchases pizza and other consumables for meetings
		\item Represents our club at E-Council
		\item Produces and Submits applications for funding
	\end{itemize}

	All of these responsibilities may be delegated to other club members, though they are ultimately the responsibility of the aforementioned officers.  The President will probably be the hardest working member of the club, and therefore needs the power to make unilateral decisions for the sake of saving time.  No officer will necessarily have any control over software projects, since by the definition of opensource software, code forks can happen when they need to, and no one therefore has ultimate power over a software project.

	\subsection{Part 2 - Officer Selection}

	Once a year, elections will take place for the officer positions.  This election must be announced at least one week prior to the vote.  Any voting member who meets The Ohio State University's requirements may run for any of the offices.  The individual who receives the most votes will attain that position for the next academic year, unless removed.  Ties will be broken by a vote of the officers in place before the active election.

	\subsection{Part 3 - Officer Removal}

	Leadership is needed in any club and so officers are required to be
present at eight out of ten official meetings per quarter, while it is also
realized that extenuating circumstances are possible.  In order to eliminate
doubts surrounding an officers excuse for missing a meeting, said officer
must notify at least one of the other officers that they will not be in
attendance at least 24 hours in advance.  If an officer is in violation of
the aforementioned attendance rules, then the club members will take a vote
as to whether or not said officer will be allowed to keep his or her
position. For all other cases of misconduct, the officer will face the same
removal process as any other member of the club.

	\section{Article IV - Advisor: Qualification Criteria}

	The advisor should be a person of technical experience. Preferably someone who has been involved in Opensource development. The advisor must grok the goals of the club so that he or she does not get in their way. The advisor exists to provide guidance, mentor-ship, and cluefulness. The advisor's term is one academic year.

	University guidelines demand that the advisor (or failing that, the sub-advisor) ``of student organizations must be full-time members of the University faculty or Administrative \& Professional staff.''

	\section{Article V - Meetings of the Organization}

%	\sout{The organization should meet however frequently it needs to in order to ``Get Things Done.'' Meetings can be formal or informal, can be in person or over an electronic medium. The current model is that regular organizational meetings are scheduled for every two weeks, and one or two formal meetings are held every quarter. These ``formal'' meetings are the required meetings for voting membership.}

	It should be recognized that those involved in individual software projects will have meetings separate from the meetings of rest of the club. It is encouraged, however, that all club members be invited to these meetings.

	If none of the officers of the club can be in attendance at a given meeting then said meeting will not take place, as decisions cannot be made without the presiding officers. Members can still hold an independent meeting, but it will not be recognized as an official club meeting.

	\section{Article VI - Method of Amending Constitution: Proposals, notices and voting requirements}

	In the case that someone thinks that they need to alter the constitution or by-laws (if any) for bug fixes, adding features, or compatibility with new hardware, a quorum (3/4 of the voting membership) must vote and the vote must be a 2/3 majority. Voting can be in person or via some electronic medium provided that in either case a reasonable amount of certainty of identity can be secured. Amendments and changes should be taken advisedly and considered for a reasonable amount of time before being implemented.

	It is strictly discouraged that the constitution should be amended frequently. If it is to be amended, and by-laws do not exist, this this article (article VI) allows the creation of by-laws in favor of amending the main body of the constitution (this document).

	It is further strictly discouraged that any amendments or by-laws restricting free commerce and creativity be created. New amendments or by-laws should not conflict with the above stated purpose of the club.

	\section{Article VII - Method of Dissolution of Organization}

	Should the club be forced to dissolve itself, any and all assets should be put toward the club's debt (if any). This includes university owned equipment or university funds.

	The remaining assets (hardware, operating funds, etc) should be donated to a free software organization, like the fsf. 5 The organization to which the assets are donated must be determined at the time of dissolution by the Benevolent Dictator, with approval of the supervisor.

\end{document}
